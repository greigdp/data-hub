\documentclass[11pt, a4paper, oneside, english]{scrbook}

\usepackage[english]{babel}
\usepackage{blindtext}

% Set fonts.
\usepackage{fontspec,xltxtra,xunicode}
\defaultfontfeatures{Mapping=tex-text}
\setromanfont[Scale=MatchLowercase, Mapping=tex-text]{Alegreya}
\setsansfont[Scale=MatchLowercase, Mapping=tex-text]{Candela Book}
\setmonofont[Scale=MatchLowercase]{Menlo}

% Set Margins for pdf copy.
\usepackage[left=7em, right=21em, top=6.5em, bottom=11em]{geometry}

% Set text ragged right.
\usepackage{ragged2e}
\RaggedRight

% Prevent over-eager hyphenation.
\hyphenpenalty=5000
\tolerance=1000

% Set Line spacing.
\usepackage{setspace}
\setstretch{1.1} % Spaces Alegreya nicely

% Set paragraph indents to 1.5 em.
\setlength{\parindent}{1em}

% No space between paragraphs.
\setlength{\parskip}{0ex}

% No bold headings.
\usepackage{sectsty}
\allsectionsfont{\mdseries}

% Set page numbers and tweak pages at the start of chapters.
% NB: fancyhdr *has* to load after the geometry package!
\usepackage{fancyhdr}
\pagestyle{fancy}
\fancyhf{}
\fancyfoot[R]{\thepage}
\fancypagestyle{plain}{\fancyfoot[R]{\thepage}}

% Force width of header rule to zero.
\renewcommand{\headrulewidth}{0pt}
\renewcommand{\footrulewidth}{0pt}

% Colours
\usepackage[usenames,dvipsnames,svgnames,table]{xcolor}
\definecolor{light-grey}{gray}{0.80}

% Set code listings with the correct spacing
\newcommand{\lstsinglespacing}{
  \setstretch{0.95} % Tighten up Menlo line spacing
}
\usepackage{listings}
\lstset{basicstyle=\lstsinglespacing\footnotesize\ttfamily,
        showstringspaces=false,
        keepspaces=true,
        tabsize=2,
        commentstyle=\color{gray},
        captionpos=t,
        belowcaptionskip=1.5ex,
        frame=lines,
        numbers=left,
        numberstyle=\footnotesize\ttfamily\color{light-grey},
        numbersep=1em}

% remove default keywords for XML and highlight HTML style comments.
% [Fix;me: add Android-specific syntax highlighting.]
\lstdefinelanguage{XML}
{
  language=html,
  basicstyle=\lstsinglespacing\footnotesize\ttfamily,
  morestring=[b]",
  morecomment=[s]{<?}{?>},
  morecomment=[s][\color{gray}]{<!--}{-->},
  keywordstyle=\normalfont\ttfamily
}

% Captions left justified
\usepackage[format=plain,
            labelsep=newline,
            singlelinecheck=false,
            justification=raggedright,
            font={normalsize, normalfont},
            labelfont={normalsize, bf}
           ]{caption}


% ************************************************************************* %

\begin{document}
\chapter{Data Hub: The Android App} % (fold)
\label{cha:data_hub}
[Fix;me: Intro to Android app. Main activity\lstinline{Hub.java}. Class to set up and manage Bluetooth connections\lstinline{BluetoothLinkService.java}.]
\section{Receiving Bluetooth Data} % (fold)
\label{sec:receiving_bluetooth_data}
[Fix;me: add an overview of the section here and/or remove the following subsection heading.]
\subsection{Initialising a Connection} % (fold)
\label{sub:initialising_a_connection}
Before initialising a connection some requirements must be satisfied in order that the app is able to receive data over the Bluetooth. The first is that the MAC address of the slave device sending the data must be known. Therefore the MAC address is hard coded as the constant\lstinline{MAC_ADDRESS} in the main activity of the app. In the example shown in this documentation the MAC address of the test embedded system is 00:12:06:82:84 [Fix;me: update MAC address]. Future iterations of the design of the app should allow this parameter to be configurable. Details of how this might be achieved without risking losing connectivity, and therefore data, are given in Section~\ref{sec:configurable_bluetooth_mac_adress}.

Secondly, before attempting to set up a Bluetooth connection, the app must establish that the device on which it is installed can support Bluetooth and that it has been enabled. Both of these tasks are performed in the\lstinline{onCreate()} lifecycle callback method. If Bluetooth is not supported the user is alerted using a toast message and the app stops running. If the local Bluetooth adapter is available it is enabled---if it is not already---by prompting the user to do so. The app is now ready to attempt to create a connection. 

\begin{lstlisting}[language=Java, numbers=none]
@Override
public void onCreate(Bundle savedInstanceState)
{
  // ...

  mBluetoothAdapter = BluetoothAdapter.getDefaultAdapter();
  if (mBluetoothAdapter == null)
  {
    Toast.makeText(this,
                   "Bluetooth is not supported",
                   Toast.LENGTH_LONG).show();
    finish();
    return;
  }
  else
  {

    if (!mBluetoothAdapter.isEnabled())
      Intent enableIntent = new Intent(BluetoothAdapter.
                                       ACTION_REQUEST_ENABLE);
    else
      // Set up the link session.
  }
}
\end{lstlisting}
The main activity of the app uses two methods to initiate a connection:\lstinline{setupLink()} and\lstinline{connectDevice()}. Each performs a simple function. A new instance of\lstinline{BluetoothLinkService} is created when\lstinline{setupLink()} is called. Then, when\lstinline{connectDevice()} is called, a\lstinline{BluetoothDevice} object representing the slave device is instantiated using\lstinline{MAC_ADDRESS} which is passed as an argument in a call of the\lstinline{connect()} method to the\lstinline{BluetoothLinkService} instance created in\lstinline{setupLink()}.
\begin{lstlisting}[language=Java,
                   numbers=none]
pivate void setupLink()
{ 
  mLinkService = new BluetoothLinkService(this, mHandler);
}

private void connectDevice()
{
  BluetoothDevice device = mBluetoothAdapter.getRemoteDevice(MAC_ADDRESS);
  mLinkService.connect(device);
}
\end{lstlisting}
% subsection initiiate_a_connection (end)
\subsection{The\lstinline{BluetoothLinkService} Class} % (fold)
\label{sub:bls}
A\lstinline{BluetoothLinkService} is created by passing an application context and handler to the constructor.
\begin{lstlisting}[language=Java, numbers=none]
public BluetoothLinkService(Context context, Handler handler)
{
  mContext = context;
  mHandler = handler;
  mAdapter = BluetoothAdapter.getDefaultAdapter();
  // ...
}
\end{lstlisting}
When the constructor is called in\lstinline{Hub.java} (see\lstinline{setupLink()}) the\lstinline{this} keyword is used to pass the pass the context of the main activity, and a new handler is created to receive messages sent from the threads used by the\lstinline{BluetoothLinkService}. A\lstinline{BluetoothAdapter} is also instantiated to represent the default adapter of the device.

\lstinline{mHandler} is primarily used to update the user interface on changes in status of the Bluetooth connection and is discussed in detail in Sub-section~\ref{sec:the_user_interface}. The remainder of this Sub-section details the operation of\lstinline{BluetoothLinkService} class.

The\lstinline{BluetoothLinkService} class sets up and manages Bluetooth connections. It uses two separate threads to connect to devices, and when connected, to receive data. These threads are controlled by three methods:
\begin{enumerate}
\item\lstinline{connect()} to initiate a connection to a remote (slave) device,
\item\lstinline{connected()} to manage a\lstinline{BluetoothLinkService} which is running a\lstinline{ConnectedThread}, and
\item\lstinline{stop()} to stop all threads running as part of a\lstinline{BluetoothLinkService}.
\end{enumerate}
The\lstinline{connect()} method closes any open connections, including those currently connecting, and then creates a new\lstinline{ConnectThread} to establish a connection. When called\lstinline{ConnectThread} instantiates a local\lstinline{BluetoothSocket} using the default secure serial port protocol UUID, and a\lstinline{BluetoothDevice} using the parameter passed to the\lstinline{connect()} method. If this is successful the thread runs.

Before trying to connect, any service discovery which is running on the adapter is cancelled to prevent it from slowing the connection. The\lstinline{connect()} method is then called on the socket. If a connection is established successfully, the thread is reset and the\lstinline{connected()} method is called. The connection is now established.
\begin{lstlisting}[language=Java, numbers=none]
public synchronized void connect(BluetoothDevice device)
{
  // ...
  mConnectThread = new ConnectThread(device);
  mConnectThread.start();
  // ...  
}

private class ConnectThread extends Thread
  {
    private final BluetoothSocket mSocket;
    private final BluetoothDevice mDevice;

    public ConnectThread(BluetoothDevice device)
    {
      mDevice = device;
      BluetoothSocket tmp = null;

      try
      {
        tmp = device.createRfcommSocketToServiceRecord(SECURE_UUID);
      }
      catch (IOException e)
      {
        Log.e(TAG, "Failed to create socket.\n", e);
      }
      mSocket = tmp;
    }

    public void run()
    {
      mAdapter.cancelDiscovery();

      try
      {
        if (Debug) Log.i(TAG, "Trying to connect.");
        mSocket.connect();
      }
      catch (IOException e)
      {
        // ...
      }

      synchronized (BluetoothLinkService.this)
      {
        mConnectThread = null;
      }

      connected(mSocket, mDevice);
    }

    // cancel()
  }
\end{lstlisting}
The\lstinline{connected()} method creates a new\lstinline{ConnectedThread} and starts it. The\lstinline{ConnectedThread} creates a local\lstinline{BluetoothSocket} from the instance passed as an argument also creates a local\lstinline{InputStream}. If an\lstinline{InputStream} is available form the socket the thread runs.

\lstinline{ConnectedThread} continually listens to the\lstinline{InputStream} while connected. A\lstinline{BufferedReader} is created and each array of bytes sent over the Bluetooth connection is read into it using an\lstinline{InputStreamReader}. Any data in the\lstinline{BufferedReader} it is parsed, sent to the user interface for display and stored locally on the device.
\begin{lstlisting}[language=Java, numbers=none]
public synchronized void connected(BluetoothSocket socket,
                                    BluetoothDevice device)
{ 
  // Cancel any threads already running here.

  mConnectedThread = new ConnectedThread(socket);
  mConnectedThread.start();

  // ...
}

private class ConnectedThread extends Thread
{
  private final BluetoothSocket mSocket;
  private final InputStream mInStream;  

  public ConnectedThread(BluetoothSocket socket)
  {
    mSocket = socket;
    InputStream tmpIn = null;  

    try
    {
      tmpIn = socket.getInputStream();
    }
    catch (IOException e)
    {
      // ...
    }  

    mInStream = tmpIn;
  }

  public void run()
  {
    while (true)
    {
      try
      {
        BufferedReader reader = new BufferedReader(
          new InputStreamReader(mInStream));
        String line;  

        while ((line = reader.readLine()) != null)
        {
          // Save data to DB and send to UI.
        }
      }
      catch (IOException e)
      {
        // ...
      }
    }
  }

  // cancel()
}
\end{lstlisting}
% subsection bluetoothlinkservice (end)
% section receiving_bluetooth_data (end)
\section{Gathering Location Data} % (fold)
\label{sec:gathering_location_data}
[Fix;Me: fold in location gathering code and test.]
% section gathering_location_data (end)
\section{The User Interface} % (fold)
\label{sec:the_user_interface}
[Fix;Me: document the\lstinline{Handler} in\lstinline{Hub.java}.]
% section the_user_interface (end)
\section{Storing Data Locally} % (fold)
\label{sec:storing_data_locally}
Data is stored locally in an SQLite database on the SD card of the device. Each line of data that is received via Bluetooth is parsed and stored by the\lstinline{BluetoothLinkService}. Comma-separated values are saved as elements of an array list and local variables representing each sample value are instantiated. A new\lstinline{HubDdHelper} is then created along with a new\lstinline{AccelerometerWrapper}, each of the samples values is added to the \lstinline{AccelerometerWrapper}, and the data is added to the database by passing the \lstinline{AccelerometerWrapper} as an argument to a call to\lstinline{addAccSample()}.
\begin{lstlisting}[language=Java,
                   numbers=none]
while ((line = reader.readLine()) != null)
{
  List<String> data = Arrays.asList(line.split(","));
  if (data.size() == 7)
  {
    if (Debug) Log.i(TAG, "Received Data: " + line);  

    String timestamp = data.get(0);
    int xaxis = Integer.parseInt(data.get(2));
    int yaxis = Integer.parseInt(data.get(3));
    int zaxis = Integer.parseInt(data.get(4));

    // Temperature data...

  HubDbHelper db = new HubDbHelper(mContext);
  AccelerometerWrapper accWrap = 
    new AccelerometerWrapper(timestamp, xaxis, yaxis, zaxis);
  db.addAccSample(accWrap);
}
\end{lstlisting}
When a\lstinline{HubDbHelper} is instantiated, by passing the application context\lstinline{mContext} to the constructor (which was passed to the\lstinline{BluetoothLinkService} from the main activity), it constructs a sting representation of the file path to the database and uses the application context and file path to create a\lstinline{HubDbOpenHelper}.
\begin{lstlisting}[language=Java, numbers=none]
private final static String DB_NAME = "hub_db.sqlite3";
private final static String BASE_DIR = "Hub";

public HubDbHelper(Context context)
{
  String dbPath = HubDbHelper.getDbPath() + "/" + DB_NAME;
  dbHelper = new HubDbOpenHelper(context, dbPath);
}

private static String getDbPath()
{
  String path = Environment.getExternalStorageDirectory()
                           .getPath() + "/" + BASE_DIR;

  File dbDir = new File(path);
  if (!dbDir.isDirectory())
  {
    try
    {
      if (Debug) Log.i(TAG, "Trying to create " + path);
      dbDir.mkdirs();
    }
    catch (Exception e)
    {
      final Writer result = new StringWriter();
      final PrintWriter printWriter = new PrintWriter(result);
      e.printStackTrace(printWriter);
      Log.e(TAG, result.toString());
    }
  }
  return path;
}
\end{lstlisting}
The\lstinline{HubDbOpenHelper} class extends\lstinline{SQLiteOpenHelper} and either opens a database located at the file path directory passed to the constructor, or creates a database at the file path directory if one does not yet exist. The database is created by overriding the \lstinline{onCreate()} method of the helper class: an SQLite database is passed as an argument, and the SQL query required to create the database is then passed to a call to the\lstinline{execSQL()} method.
\begin{lstlisting}[language=Java, numbers=none]
public final static String ACC_TABLE_NAME = "accelerometer_data";
public final static String ID = "id";
public final static String TIMESTAMP = "timestamp";
public final static String X_AXIS = "xaxis";
public final static String Y_AXIS = "yaxis";
public final static String Z_AXIS = "zaxis";

public HubDbOpenHelper(Context context, String path)
{
  super(context, path, null, 1);
}

@Override
public void onCreate(SQLiteDatabase db)
{
  String createAccSQL = "create table "
                      + ACC_TABLE_NAME
                      + " ("
                      + ID + " integer primary key autoincrement, "
                      + TIMESTAMP + " text, "
                      + X_AXIS + " integer, "
                      + Y_AXIS + " integer, "
                      + Z_AXIS + " integer);";
  db.execSQL(createAccSQL);
}
\end{lstlisting}
The\lstinline{AccelerometerWrapper} class provides an abstraction of the data contained in each accelerometer sample. It is a simple class. Each datum is passed to the constructor as an argument and a corresponding instance variable is set according to the value of the argument. These instance variables are private and can not be altered after they are set by the constructor. Accessor methods are available to retrieve the values of each datum in the accelerometer sample.
\begin{lstlisting}[language=Java, numbers=none]
public class AccelerometerWrapper
{
  private String timestamp;
  private int xaxis;
  private int yaxis;
  private int zaxis;

  public AccelerometerWrapper(String t, int x, int y, int z)
  {
    this.timestamp = t;
    this.xaxis = x;
    this.yaxis = y;
    this.zaxis = z;
  }

  public String getTimestamp()
  {
    return timestamp;
  }

  // Get x axis, y axis, and z axis too.
}
\end{lstlisting}
The\lstinline{addAccSample()} method gets a writable database, adds the data to a\lstinline{ContentValues} object and attempts to add this object to the database. If the insertion is successful the database is closed or, if there is a problem, an\lstinline{SQLExeption} is thrown.
\begin{lstlisting}[language=Java, numbers=none]
  public synchronized void addAccSample(AccelerometerWrapper accWrap)
  {
    SQLiteDatabase db = null;
    try
    {
      db = dbHelper.getWritableDatabase();
      ContentValues values = new ContentValues();
      values.put(HubDbOpenHelper.TIMESTAMP, accWrap.getTimestamp());
      values.put(HubDbOpenHelper.X_AXIS, accWrap.getX());
      values.put(HubDbOpenHelper.Y_AXIS, accWrap.getY());
      values.put(HubDbOpenHelper.Z_AXIS, accWrap.getZ());
      db.insertOrThrow(HubDbOpenHelper.ACC_TABLE_NAME,
                       HubDbOpenHelper.TIMESTAMP,
                       values);
    }
    catch(SQLException e)
    {
      Log.e(TAG, "Could not insert record:\n" + e);
    }
    finally
    {
      if (db != null) db.close();
    }
  }
\end{lstlisting}
% section storing_data_locally (end)
\section{Storing Data Remotely} % (fold)
\label{sec:storing_data_remotely}

\subsection{Syncing Local Data with a Remote Database} % (fold)
\label{sub:syncing_local_data_with_a_remote_database}
\begin{lstlisting}[language=Java, numbers=none]
@Override
public void onStart()
{
  // ...
  startGatheringData();
}
\end{lstlisting}
\begin{lstlisting}[language=Java, numbers=none]
@Override
public int onStartCommand(Intent intent, int flags, int startId)
{
  // ...
  mAlarmManager = (AlarmManager) getSystemService(Context.ALARM_SERVICE);
  startSync();
  //...
}

@Override
public void onDestroy()
{
  // ...
  stopSync();
}

private void startSync()
{
  mSyncIntent = 
    new Intent(getApplicationContext(), SyncAlarmReceiver.class);
  mSyncPendingIntent = 
    PendingIntent.getBroadcast(getApplicationContext(), 0, mSyncIntent, 0);
  mAlarmManager.setInexactRepeating(AlarmManager.ELAPSED_REALTIME_WAKEUP,
                                    SystemClock.elapsedRealtime(),
                                    SYNC_PERIOD,
                                    mSyncPendingIntent);
}

private void stopSync()
{
  if (mSyncPendingIntent != null)
    mAlarmManager.cancel(mSyncPendingIntent);
}
\end{lstlisting}
\begin{lstlisting}[language=Java, numbers=none]
@Override
public void onReceive(Context context, Intent intent)
{
  Intent syncIntent = new Intent(context, SyncService.class);
  context.startService(syncIntent);
}
\end{lstlisting}
\begin{lstlisting}[language=Java, numbers=none]
private PowerManager mPowerManager;
private PowerManager.WakeLock mWakeLock;

@Override
public void onCreate()
{
  super.onCreate();
    
  mPowerManager = (PowerManager) getSystemService(Context.POWER_SERVICE);
  mWakeLock =
    mPowerManager.newWakeLock(PowerManager.PARTIAL_WAKE_LOCK, TAG);
  mWakeLock.acquire();
}
\end{lstlisting}
\begin{lstlisting}[language=Java, numbers=none]
private SyncThread mSyncThread;

@Override
public int onStartCommand(Intent intent, int flags, int startId)
{
  mSyncThread = new SyncThread();
  mSyncThread.start();

  return 0;
}
\end{lstlisting}
\begin{lstlisting}[language=Java, numbers=none]
private static final int CONNECTION_TIMEOUT = 10000;
private static final String LATEST_ACC_PATH =
    "http://sederunt.org/movements/latest";
private static final String ACC_PATH =
    "http://sederunt.org/movements/input";
private static final String CHARSET = "UTF-8";

private class SyncThread extends Thread
{
  public SyncThread()
  {
    // Empty constructor?
  }

  public void run()
  {
    String latestAccId = "";
    DefaultHttpClient httpClient =
        new DefaultHttpClient(new BasicHttpParams());
      HttpConnectionParams.setConnectionTimeout(httpClient.getParams(),
                                                CONNECTION_TIMEOUT);

    try
    {
      HttpGet getLatestAcc = new HttpGet(LATEST_ACC_PATH);
      HttpResponse latestAcc = httpClient.execute(getLatestAcc);
      int statusCode = latestAcc.getStatusLine().getStatusCode();

      if(statusCode == HttpStatus.SC_OK)
      {
        HttpEntity httpEntity = latestAcc.getEntity();
        if(httpEntity != null)
        {
          latestAccId = EntityUtils.toString(httpEntity,
                                             SyncService.CHARSET);
          latestAccId = latestAccId.replace("\n", "").replace("\r", "");
        }
      }
    }
    catch (Exception e)
    {
      Log.e(SyncService.TAG, "Exception occured: " + e.getMessage());
    }

    HubDbHelper db = new HubDbHelper(SyncService.this);

    ArrayList dataList =
      db.getLatestMovement(Integer.parseInt(latestAccId));
    for (Object data : dataList)
    {
      updateServer(ACC_PATH, data.toString());
    }
  }

  public void cancel()
  {
    // Empty method?
  }
}
\end{lstlisting}
\begin{lstlisting}[language=Java, numbers=none]
private int updateServer(String url, String data)
{    
  HttpParams mParams = new BasicHttpParams();
  HttpConnectionParams.setConnectionTimeout(mParams, CONNECTION_TIMEOUT);
  HttpConnectionParams.setSoTimeout(mParams, CONNECTION_TIMEOUT);
  HttpClient httpClient = new DefaultHttpClient(mParams);

  try
  {
    HttpPost httpPost = new HttpPost(url);
    httpPost.setHeader("Content-type", "application/json");

    StringEntity stringEntity = new StringEntity(data);
    stringEntity.setContentEncoding(new BasicHeader(HTTP.CONTENT_TYPE,
                                      "application/json"));
    httpPost.setEntity(stringEntity);

    HttpResponse response = httpClient.execute(httpPost);
    HttpEntity entity = response.getEntity();
    StatusLine status = response.getStatusLine();
    int statusCode = status.getStatusCode();

    if (statusCode == HttpStatus.SC_OK)
    {
      return 0;
    }
    else
    {
      return -1;
    }
  catch (Exception e)
  {
    Log.e(TAG, "Exception occured. \n" + e.getMessage());

    return -1;
  }
}
\end{lstlisting}
\begin{lstlisting}[language=Java, numbers=none]
@Override
public void onDestroy()
{
  super.onDestroy();

  mWakeLock.release();
}
\end{lstlisting}
\begin{lstlisting}[language=Java, numbers=none]
public ArrayList getLatestMovement(int latestId)
{
  ArrayList accData = new ArrayList();
  SQLiteDatabase db = null;
  int number_of_rows = 0;
  int batch_size = BATCH_SIZE;
  int number_of_batches = 0;

  try
  {
    db = dbHelper.getReadableDatabase();
    Cursor c = db.query(HubDbOpenHelper.ACC_TABLE_NAME,
                        null,
                        HubDbOpenHelper.ID + " > " + latestId,
                        null,
                        null,
                        null,
                        HubDbOpenHelper.ID + " ASC");

    number_of_rows = c.getCount();
    number_of_batches = number_of_rows / BATCH_SIZE;

    for (int b = 0; b < number_of_batches; b = b + 1)
    {
      JSONArray batch = new JSONArray();
        
      Cursor bq = 
        db.query(HubDbOpenHelper.ACC_TABLE_NAME,
                 null,
                 HubDbOpenHelper.ID + " BETWEEN " + 
                   ((b * BATCH_SIZE) + latestId) + 
                   " AND " + (((b + 1) * BATCH_SIZE) + latestId - 1),
                 null,
                 null,
                 null,
                 HubDbOpenHelper.ID + " ASC");

      while(bq.moveToNext())
      {
        int id = bq.getInt(0);
        String t = bq.getString(1);
        int x = bq.getInt(2);
        int y = bq.getInt(3);
        int z = bq.getInt(4);

        JSONObject data = new JSONObject();

        try
        {
          data.put(HubDbOpenHelper.ID, id);
          data.put(HubDbOpenHelper.TIMESTAMP, t);
          data.put(HubDbOpenHelper.X_AXIS, x);
          data.put(HubDbOpenHelper.Y_AXIS, y);
          data.put(HubDbOpenHelper.Z_AXIS, z);
          batch.put(data);
        }
        catch (org.json.JSONException e)
        {
          Log.e(TAG, "Exception occured.\n" + e.getMessage());
        }
      }
      accData.add(batch);
    }
    return accData;
  }
  finally
  {
    if (db != null) db.close();
  }
}
\end{lstlisting}
% subsection syncing_local_data_with_a_remote_database (end)
\subsection{The Sinatra App} % (fold)
\label{sub:the_sinatra_app}
[Fix;me: Sinatra app here.]
% subsection the_sinatra_app (end)
% section storing_data_remotely (end)
% chapter data_hub (end)
\chapter{Testing} % (fold)
\label{cha:testing}
\section{Accelerometer Data} % (fold)
\label{sec:accelerometer_data}
[Fix;Me: basic ``12 point'' test. Discuss calibration in Section~\ref{sec:calibration}.]
% section accelerometer_data (end)
\section{Receiving and Storing Data} % (fold)
\label{sec:receiving_data}
[Fix;me: three tests:
\begin{itemize}
  \item comparison of first n records,
  \item comparison of n randomly selected records, and
  \item trial deployment.
\end{itemize}
]
% section receiving and storing_data (end)
\section{Syncing Stored Data with a Server} % (fold)
\label{sec:syncing_stored_data_with_a_server}

% section syncing_stored_data_with_a_server (end)
% chapter testing (end)
\chapter{Further Work} % (fold)
\label{cha:further_work}
\section{Configurable Bluetooth MAC Address} % (fold)
\label{sec:configurable_bluetooth_mac_adress}
[Fix;Me: Clinicians should be able to pair an app/patient with a specific embedded system. More complex than a added a menu because we don't want patients to be able to alter the settings.]
% section configurable_bluetooth_mac_adress (end)
\section{Calibration} % (fold)
\label{sec:calibration}
[Fix;me: there are two problems which must be solved:
\begin{enumerate}
\item establishing the differences between the reference frame of the accelerometer, the embedded system, and the prosthesis for any given measurement, (All three could move.), and
\item recalibrating the accelerometer to account for drift, possibly with a ``quasi-static movements detector''.
\end{enumerate}]
% section calibration (end)
% chapter further_work (end)
\chapter{Code Listings} % (fold)
\label{cha:code_listings}
[Fix;me: Intro to code listings.]
\section{Android App} % (fold)
\label{sec:android_app}
[Fix;me: Overview of Android app code.]
\lstinputlisting[language=XML,
                 caption=\ttfamily{AndroidManifest.xml},
                 label=lst:AndroidManifestxml,
                ]{../AndroidManifest.xml}
\newpage
\lstinputlisting[language=Java,
                 caption=\ttfamily{Hub.java},
                 label=lst:Hubjava
                ]{../src/com/strath/hub/Hub.java}
\newpage
\lstinputlisting[language=Java,
                 caption=\ttfamily{BluetoothLinkService.java},
                 label=lst:BluetoothLinkServicejava,
                ]{../src/com/strath/hub/BluetoothLinkService.java}
\newpage
\lstinputlisting[language=Java,
                 caption=\ttfamily{HubDbHelper.java},
                 label=lst:hubdbhelperjava,
                ]{../src/com/strath/hub/HubDbHelper.java}
\newpage
\lstinputlisting[language=Java,
                 caption=\ttfamily{HubDbOpenHelper.java},
                 label=lst:hubdbopenhelperjava,
                ]{../src/com/strath/hub/HubDbOpenHelper.java}
\newpage
\lstinputlisting[language=Java,
                 caption=\ttfamily{DataGatheringService.java},
                 label=lst:datagatheringservicejava,
                ]{../src/com/strath/hub/DataGatheringService.java}
\newpage
\lstinputlisting[language=Java,
                 caption=\ttfamily{SyncAlarmReceiver.java},
                 label=lst:syncalarmreceiverjava,
                ]{../src/com/strath/hub/SyncAlarmReceiver.java}
\newpage
\lstinputlisting[language=Java,
                 caption=\ttfamily{SyncService.java},
                 label=lst:syncservicejava,
                ]{../src/com/strath/hub/SyncService.java}
\newpage
\lstinputlisting[language=XML,
                 caption=\ttfamily{main.xml},
                 label=lst:mainxml,
                ]{../res/layout/main.xml}
\lstinputlisting[language=XML,
                 caption=\ttfamily{styles.xml},
                 label=lst:stylesxml,
                ]{../res/values/styles.xml}
\lstinputlisting[language=XML,
                 caption=\ttfamily{strings.xml},
                 label=lst:stringsxml,
                ]{../res/values/strings.xml}
% section android_app (end)
% chapter code_listings (end)
\end{document}
